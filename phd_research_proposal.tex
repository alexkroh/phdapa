

\title{PhD research proposal\\ ``Hardware acceleration of micro-kernels for real-time applications''}
\author{Alex Kroh \\ alex.kroh@nicta.com.au}

\author{\begin{tabular}{r@{ }l} 
Candidate:      & Alexander Kroh \\[1ex] 
Supervisors: & Oliver Diessel \\
             & Gernot Heiser
\end{tabular}}
\date{\today}

\documentclass[10pt]{article}

\addtolength{\textheight}{4cm}
\addtolength{\topmargin}{-2cm}

\begin{document}
\maketitle

% Oliver:
% * What problem(s) are you trying to solve?
% * Why are these important?
% * What other work has been done in the area?
% * What is your plan of attack?
%     Can you give us a hunch of what you might expect?
%     How will the research proceed? e.g. build a model to identify applicable operating points; 
%     implement & tests using the relatively recent generation of tightly-coupled CPU-FPGA systems
% * What do you expect the outcomes of your research to be?

% Web:
% a title
% a statement of the research problem and its significance
% an outline of the method to be used to analyse the problem
% the names of any academics you have contacted in the School
% details of previous publications and/or research undertaken in your nominated area of interest


% more energetic/focused treatment
%\cite{aud_prefetch}
% \cite{aud_ada}
% \cite{luppold}
%\section{Introduction}
Safety critical embedded systems, such as automotive air bag and braking systems, require a thorough
analysis of Worst Case Execution Time (WCET) to ensure adherence to strict timing requirements.
Performance enhancement features of modern CPUs and cache warmth lead to non-determinism in execution
time and result in a pessimistic estimate for WCET. The system engineer must over provision processing
resources to ensure that, in the unlikely event that the WCET is ever observed, the system will still
have sufficient processing resources to meet timing requirements and operate in a safe manner. For
this reason, processing resources are typically dramatically underutilised in safety critical embedded
systems.

%\section{Related word}
Hardware acceleration of the core software architecture can not only improve the WCET of sequential
programs \cite{warp, micro_hyp}, but it has also 
been shown to reduce the degree of pessimism in WCET estimation \cite{aud_hardimp}.
With limited FPGA area, it may not be feasible to
accelerate the entire software system. However, I am motivated to explore the impact of accelerating a
core subset of the software system known as a micro-kernel. A micro-kernel traditionally provides a minimal
set of abstract features for resource management and access control while high level services, such as
device drivers and mission control systems, are provided by application software.
By the 90/10 rule, the kernel is not usually a candidate for hardware acceleration, but, because all
system events result in the invocation of kernel services, kernel performance has a strong correlation
with WCET.

%\section{Method}
The most frequent kernel invocations in a static system are caused by the scheduler and Inter-Process
Communication (IPC). I intend to begin my research by analysing the WCET benefits of accelerating these
features in programmable logic. The evaluation will initially be conducted through simulation and a
critical analysis of WCET. With positive results, I plan to extend the scope of research to dynamic
systems where resource creation and destruction must also be considered in the WCET estimate. Finally,
I look forward to migrating my concept to a physical safety critical system for evaluation in a real
world context.

%\section{Outcomes}
We can expect success in this research to reduce the cost, weight and energy requirements of critical
systems by improving the utilisation of processing resources. We can also expect a global improvement
in critical system safety through simplified WCET analysis. This research may also seed new research
opportunities in the application of programmable logic support for micro-kernel architectures.


\bibliography{phd_research_proposal}
%\bibliographystyle{apalike}
\bibliographystyle{unsrt}
\end{document}

