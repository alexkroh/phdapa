

\title{PhD research proposal\\ ``Hardware accelleration of micro-kernels for real-time applications''}
\author{Alex Kroh \\ alex.kroh@nicta.com.au}
\date{\today}

\documentclass[10pt]{article}

\addtolength{\textheight}{2cm}

\begin{document}
\maketitle

% Oliver:
% * What problem(s) are you trying to solve?
% * Why are these important?
% * What other work has been done in the area?
% * What is your plan of attack?
% * What do you expect the outcomes of your research to be?

% Web:
% a title
% a statement of the research problem and its significance
% an outline of the method to be used to analyse the problem
% the names of any academics you have contacted in the School
% details of previous publications and/or research undertaken in your nominated area of interest

\section{Introduction}

Safety critical embedded systems require a thorough analysis of Worst Case Execution Time (WCET)
to ensure adherence to strict safety deadlines. Performance enhancement features of modern
CPUs and cache warmth lead to non-determinism in execution time. For this reason, the WCET estimate
is pessimistic and leads to a dramatic underutilisation of processing resources.

Hardware accelleration of the core software architecture can not only improve the WCET, but it may
also reduce the degree of pessimism in the estimated value and hence improve processor utilisation.

\section{Related work}
% Need to do a lit review of FPGA based OS and kernels
% Anna @ NICTA has almost finished her 2nd year of her PhD which involves a software approach to
% improving processor utilisation
% Bernard blackham, recent PhD graduate at NICTA has critically evaluated the WCET of seL4

\section{Method}
With limited FPGA area, it would not be feasible to consider an enire software system as a candidate
for hardware accelleration. The focus is hence limited to the accelleration of an operating system
kernel. The micro kernel software architecture is ideal for this research as it traditionally provides
a minimal set of features while user level tasks are expected to provide extended features and drivers.
An operating system kernel is responsible for the following administrative tasks:

The most frequent kernel invocation events are caused by scheduling threads and IPC. We plan to begin the
research by analysing the benefits of accellerating these features in programmable logic.

\section{Outcomes}
We can expect success in this research to reduce the financial and energy costs of a critical system by
allowing the system designer to scale back on the provisioned processing power. This research will also
seed new research opportunities in the application of programmable logic to micro-kernel architectures.
One such opportunity may be the hardening of the kernel to Single Event Upsets (SEUs) through custom
Tripple Mudular Redundancy (TMR) circuits within the kernel.

\end{document}

